% !TEX root = ../main.tex
\section{Background: Variational Inference} \label{sec:bckg}

Let $\pi(x)$ be a density that we are interested in approximating.
In the Bayesian setting, $\pi(x)=p(x)/Z$
where $p(x)=\mathrm{prior}(x)\mcL(y\given x)$ combines the
prior and likelihood (given data $y$) and $Z=\int p(\tx)\dee\tx$
is the (unknown) normalizing constant (or evidence).
We assume that data have already been observed and we are just interested
in estimating $\pi$ as a function of $x$,
so we omit $y$ from notation.

Variational inference refers to approximating $\pi$
with an element $q^\star\in\mcQ=\{q_\lambda \given \lambda\in\Lambda\}$,
where $\mcQ$ is a family of probability distributions
parametrized by $\lambda$.
In the remainder of this work,
$\mcQ$ will be the family of Gaussian distributions:
$\lambda=(\mu,\Sigma)$ and
$\mcQ=\{q_\lambda=\distNorm(\mu,\Sigma) \given \mu\in\reals^d,\Sigma\geq 0\}$.
The approximation $q^\star$ is chosen to minimize some divergence
$\mathrm{D}$ from elements in $\mcQ$ to $\pi$:
\[ \label{eq:vi}
  q_\star=q_{\lambda^\star},\quad
  \lambda^\star=\argmin_{\lambda\in\Lambda}\divergence{}{q_\lambda}{\pi}.
\]
The choice of divergence can influence
the geometry of the optimization problem as well as
the characteristics of the optimal approximation $q^\star$.
Arguably, the most popular choice of divergence is
the Kullback-Leibler divergence from $q$ to $\pi$,
known as the \emph{reverse} KL:
\[ \label{eq:elbo}
  \kl[rev]{q}{\pi}=\int q(x)\log\frac{q(x)}{\pi(x)}\,\dee x
  =\int q(x)\log\frac{q(x)}{p(x)}\,\dee x+Z
  := -\mcE(q,p)+Z.
\]
In the last equation,
I factorized the evidence $Z$ out of the integral
and defined $\mcE(q,p)$ as the (negative) ``unnormalized'' KL.
$\mcE(q,p)$ is known as the Evidence Lower BOund (ELBO).
Since $Z$ does not depend on $\lambda$,
minimizing the KL divergence is equivalent to maximizing the ELBO.

We can solve this optimization problem
through the use of gradient-based optimization algorithms.
Specifically, the gradient of the ELBO is an expectation under
the variational approximation $q_\lambda$:
\[ \label{eq:elbo_grad}
  \grad_\lambda \mcE(q,p)
  =-\int q(x) \log\frac{q(x)}{p(x)} \grad_\lambda q_\lambda(x)\,\dee x.
\]
We can approximate the ELBO gradient via Monte Carlo
samples from $q$ within a stochastics gradient ascent routine
(black box vi cite).

The minimizer of the reverse KL, $q_\mathrm{rev}(x)$,
is known to underestimate the variance of $\pi(x)$
(cite ML book).
One way to address this is to minimize the \emph{forward} KL
divergence instead, i.e.,
the KL divergence from the posterior $\pi$ to the approximation:
\[
  \kl[fwd]{\pi}{q}=\int \pi(x)\log\frac{\pi(x)}{q(x)}.
\]
Common folk wisdom in the machine learning community suggests
that $\kl[fwd]{\pi}{q}$ produces better approximations,
but it is considerably more difficult to optimize.
For example, its gradient can be expressed as
an expectation but under the intractable $\pi$:
\[
  \grad_\lambda \kl[fwd]{q}{\pi}=-\int \pi(x) \grad_\lambda q(x)\,\dee x.
\]
Importance sampling can be numerically unstable for high-dimensional
$\pi$, so recent work has attempted to estimate this gradient
by running an MCMC chain in parallel to produce approximate samples from $\pi$
(cite Markovian score climbing).
