% !TEX root = ../main.tex
\section{Introduction}

The Bayesian statistical framework provides practitioners with a principled
means to obtain finite-sample uncertainty quantification guarantees.
Specifically, the posterior distribution encodes the uncertainty
around the unobserved quantities given the observations and
a prior distribution.
Posterior credible intervals (CIs) or more general regions
allow practitioners to quantify the accuracy of point estimates.


The quality of credible intervals can be measured by
the proportion of times they contain the ``true''
value of the parameter.
This is known as the frequentist coverage,
and Bayesian credible intervals are asymptotically exact
in this sense as a consequence of the Bernstein-von Mises theorem.
Another way to assess the quality of an interval is via its
\emph{Bayesian} coverage, i.e.,
the proportion of times the interval contains the parameter
value (itself sampled from the prior) that generated the data.
The Bayesian coverage of an interval leverages the fact that the prior
distribution quantifies the uncertainty around the parameter
before observing data.

In class, we showed that credible intervals attain nominal Bayesian coverage
(i.e., their coverage is equal to the credibility level) for any sample size
and without any regularity conditions.
This follows from the definition of Bayesian coverage and credible intervals.
However, this result assumes that we have access to \emph{exact}
credible intervals.
For all but the simplest of models, however,
the posterior distribution is intractable and has to be approximated numerically.
In this setting,
the resulting credible intervals are approximate
and their Bayesian coverage need not be exact.

Variational inference \citep[VI][]{jordan1999vi,wainwright2008vi}
is a scalable framework to learn posterior distributions.
Succincly, VI casts inference as an optimization problem
by approximating the posterior with an element of
a family of candidate approximations that minimizes some divergence
to the posterior.
The quality of the approximation depends on
the flexibility of the family of approximations,
the divergence being minimized,
and whether the optimization problem can be solved reliably.
A simple instantiation of VI consists on finding the best
Gaussian approximation to the posterior distribution
(a framework called Gaussian VI).
In this case, the optimization problem is usually amenable
to off-the-shelf stochastic optimization algorithms
for many common divergences.

In this work, I study the Bayesian coverage of
credible intervals of Gaussian VI when minimizing
the Kullback-Leibler divergence \citep[KL][]{kl}.
Specifically, I focus on the impact of minimizing the reverse KL
(from approximation to target) versus the forward KL (from target to approximation).
The former is known to produce approximations that underestimate
the posterior variance and viceversa.
Through two simulation studies,
I conclude that the forward KL tends to have better Bayesian coverage
when the posterior has heavier-than-Gaussian tails,
which is a common occurrence
(e.g., in Bayesian logistic regression, see \cref{subsec:logreg}.)
