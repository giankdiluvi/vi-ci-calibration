% !TEX root = ../main.tex
\section{Conclusion} \label{sec:conclusion}

In this work,
I studied the Bayesian coverage of credible intervals
obtained through Gaussian variational approximations that minimize
either the forward or the reverse KL divergence.
As expected,
the forward KL-optimal approximations $q_\mathrm{fwd}$
tend to have a higher variance than their reverse KL counterparts,
$q_\mathrm{bwd}$.
Both of the posterior distributions that I studied have tails heavier
than a Gaussian, and so naturally $q_\mathrm{fwd}$ produced intervals
with coverage closer to the nominal value $1-\alpha$.

Note that the concept of Bayesian coverage by definition
only measures the extent to which we are able to approximate
the posterior distribution.
A visual interpretation of this can be seen in \cref{fig:logreg_p}:
approximations with good Bayesian coverage are those that
approximate the posterior $\pi(x)$ well,
regardless of the ``true'' value of the parameter $\beta_1$.
A very peaky distribution centered around $\beta_1$
would likely have worse Bayesian coverage than any of the
approximations produced in that example.
This is offset by the fact that,
as the sample size grows,
the posterior concentrates around the maximum likelihood estimator,
which under regularity conditions is consistent.

Finally, in this work we used a considerably basic variational family.
More expressive families will produce better approximations
that in turn result in intervals with better Bayesian coverage,
even if we minimize the reverse instead of the forward KL divergence.
